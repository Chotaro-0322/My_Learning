% Created 2021-02-03 水 17:47
% Intended LaTeX compiler: pdflatex
\documentclass[11pt]{article}
\usepackage[utf8]{inputenc}
\usepackage[T1]{fontenc}
\usepackage{graphicx}
\usepackage{grffile}
\usepackage{longtable}
\usepackage{wrapfig}
\usepackage{rotating}
\usepackage[normalem]{ulem}
\usepackage{amsmath}
\usepackage{textcomp}
\usepackage{amssymb}
\usepackage{capt-of}
\usepackage{hyperref}
\author{itolab-chotaro}
\date{\today}
\title{}
\hypersetup{
 pdfauthor={itolab-chotaro},
 pdftitle={},
 pdfkeywords={},
 pdfsubject={},
 pdfcreator={Emacs 27.1 (Org mode 9.4.4)}, 
 pdflang={English}}
\begin{document}

\tableofcontents

\section{def async(coro\textsubscript{or}\textsubscript{future}, loop=None):}
\label{sec:org075847c}
\begin{itemize}
\item Error
>>> imoport trollius
File "<stdin>", line 1
  imoport trollius
                \^{}
SyntaxError: invalid syntax
>>> import trollius
Traceback (most recent call last):
  File "<stdin>", line 1, in <module>
  File "\emph{home/itolab-chotaro}.pyenv/versions/3.7.5/envs/roswork/lib/python3.7/site-packages/trollius/\_\textsubscript{init}\_\textsubscript{.py}", line 21, in <module>
    from .base\textsubscript{events} import *
  File "\emph{home/itolab-chotaro}.pyenv/versions/3.7.5/envs/roswork/lib/python3.7/site-packages/trollius/base\textsubscript{events.py}", line 42, in <module>
    from . import tasks
  File "\emph{home/itolab-chotaro}.pyenv/versions/3.7.5/envs/roswork/lib/python3.7/site-packages/trollius/tasks.py", line 565
    def async(coro\textsubscript{or}\textsubscript{future}, loop=None):
            \^{}
SyntaxError: invalid syntax

\item Solution
asyncとは, すでにpython3.7によって定義されているため使用できないと言っている?
別の名前にする必要があるかもしれない.

これができないと, catkinを使用できない.
どうやらインストールしたcatkin\textsubscript{toolsに問題があったみたい}?
gitから直接catkin\textsubscript{toolsをインストールする}.
\$ sudo pip3 install git+\url{https://github.com/catkin/catkin\_tools.git}
\end{itemize}

\section{\$ ./src/catkin/bin/catkin\textsubscript{make}\textsubscript{isolated} --install -DCMAKE\textsubscript{BUILD}\textsubscript{TYPE}=Release -DPYTHON\textsubscript{EXECUTABLE}=/home/itolab-chotaro/.pyenv/versions/3.7.5/bin/python3.7m -DPYTHON\textsubscript{INCLUDE}\textsubscript{DIR}=/home/itolab-chotaro/.pyenv/versions/3.7.5/include/python3.7m}
\label{sec:org492d139}
\begin{itemize}
\item Error
(省略)
-- Using Python nosetests: /usr/bin/nosetests3
ImportError: "from catkin\textsubscript{pkg.package} import parse\textsubscript{package}" failed: No module named 'catkin\textsubscript{pkg}'
Make sure that you have installed "catkin\textsubscript{pkg}", it is up to date and on the PYTHONPATH.
CMake Error at cmake/safe\textsubscript{execute}\textsubscript{process.cmake}:11 (message):
  execute\textsubscript{process}(\emph{home/itolab-chotaro}.pyenv/versions/3.7.5/bin/python3.7m
  "\emph{home/itolab-chotaro/ros/melodic/src/catkin/cmake/parse\textsubscript{package}\textsubscript{xml.py}"
  "/home/itolab-chotaro/ros/melodic/src/catkin/cmake}../package.xml"
  "/home/itolab-chotaro/ros/melodic/build\textsubscript{isolated}/catkin/catkin/catkin\textsubscript{generated}/version/package.cmake")
  returned error code 1
Call Stack (most recent call first):
  cmake/catkin\textsubscript{package}\textsubscript{xml.cmake}:74 (safe\textsubscript{execute}\textsubscript{process})
  cmake/all.cmake:168 (\textsubscript{catkin}\textsubscript{package}\textsubscript{xml})
  CMakeLists.txt:8 (include)
\end{itemize}


-- Configuring incomplete, errors occurred!
See also "/home/itolab-chotaro/ros/melodic/build\textsubscript{isolated}/catkin/CMakeFiles/CMakeOutput.log".
See also "/home/itolab-chotaro/ros/melodic/build\textsubscript{isolated}/catkin/CMakeFiles/CMakeError.log".
<== Failed to process package 'catkin': 
  Command '['cmake', '\emph{home/itolab-chotaro/ros/melodic/src/catkin', '-DCATKIN\textsubscript{DEVEL}\textsubscript{PREFIX}=/home/itolab-chotaro/ros/melodic/devel\textsubscript{isolated}/catkin', '-DCMAKE\textsubscript{INSTALL}\textsubscript{PREFIX}=/home/itolab-chotaro/ros/melodic/install\textsubscript{isolated}', '-DCMAKE\textsubscript{BUILD}\textsubscript{TYPE}=Release', '-DPYTHON\textsubscript{EXECUTABLE}=/home/itolab-chotaro}.pyenv/versions/3.7.5/bin/python3.7m', '-DPYTHON\textsubscript{INCLUDE}\textsubscript{DIR}=/home/itolab-chotaro/.pyenv/versions/3.7.5/include/python3.7m', '-G', 'Unix Makefiles']' returned non-zero exit status 1

Reproduce this error by running:
==> cd \emph{home/itolab-chotaro/ros/melodic/build\textsubscript{isolated}/catkin \&\& cmake /home/itolab-chotaro/ros/melodic/src/catkin -DCATKIN\textsubscript{DEVEL}\textsubscript{PREFIX}=/home/itolab-chotaro/ros/melodic/devel\textsubscript{isolated}/catkin -DCMAKE\textsubscript{INSTALL}\textsubscript{PREFIX}=/home/itolab-chotaro/ros/melodic/install\textsubscript{isolated} -DCMAKE\textsubscript{BUILD}\textsubscript{TYPE}=Release -DPYTHON\textsubscript{EXECUTABLE}=/home/itolab-chotaro}.pyenv/versions/3.7.5/bin/python3.7m -DPYTHON\textsubscript{INCLUDE}\textsubscript{DIR}=/home/itolab-chotaro/.pyenv/versions/3.7.5/include/python3.7m -G 'Unix Makefiles'

Command failed, exiting.

\begin{itemize}
\item Solution
PYTHONPATHにcatkin\textsubscript{pkgが入っていないのが原因}.
今回の場合, pyenv環境なのでcatkin\textsubscript{pkgの場所を確認してPYTHONPATHに追加}.

\$ export PYTHONPATH="\emph{home/itolab-chotaro}.pyenv/versions/3.7.5/envs/roswork/lib/python3.7/site-packages:\$PYTHONPATH"
\end{itemize}

\section{\$ ./src/catkin/bin/catkin\textsubscript{make}\textsubscript{isolated} --install -DCMAKE\textsubscript{BUILD}\textsubscript{TYPE}=Release -DPYTHON\textsubscript{EXECUTABLE}=/home/itolab-chotaro/.pyenv/versions/3.7.5/bin/python3.7m -DPYTHON\textsubscript{INCLUDE}\textsubscript{DIR}=/home/itolab-chotaro/.pyenv/versions/3.7.5/include/python3.7m}
\label{sec:org9591c34}
\begin{itemize}
\item Error
Makefile:94: recipe for target 'install' failed
make: \textbf{*} [install] Error 1
<== Failed to process package 'catkin': 
  Command '['make', 'install']' returned non-zero exit status 2

Reproduce this error by running:
==> cd /home/itolab-chotaro/ros/melodic/build\textsubscript{isolated}/catkin \&\& make install

Command failed, exiting.

\item Try
makeが認識されていない? or 入っていない.
\$ which make
> /usr/bin/make
になっていた. pyenv環境にインストールしたほうが良い?
\$ pip3 install make
によってmakeを仮想環境内にインストール
しかしこの問題は解決しなかった.

\item Solution
\$ ./src/catkin/bin/catkin\textsubscript{make}\textsubscript{isolated} --install の項目に
-DSETUPTOOLS\textsubscript{DEB}\textsubscript{LAYOUT}=OFF を追加することで解決できるらしい.

しかしこの方法は一時的な解決策であるため, あまり良くないような気がする(ネットの意見)
\end{itemize}

\section{\$ ./src/catkin/bin/catkin\textsubscript{make}\textsubscript{isolated} --install -DCMAKE\textsubscript{BUILD}\textsubscript{TYPE}=Release -DPYTHON\textsubscript{EXECUTABLE}=/home/itolab-chotaro/.pyenv/versions/3.7.5/bin/python3.7m -DPYTHON\textsubscript{INCLUDE}\textsubscript{DIR}=/home/itolab-chotaro/.pyenv/versions/3.7.5/include/python3.7m}
\label{sec:orgc775d1b}
\begin{itemize}
\item Error
Unable to find the requested Boost libraries.

Boost version: 1.58.0

Boost include path: /usr/include

Could not find the following Boost libraries:

boost\textsubscript{thread}
boost\textsubscript{locale}

Some (but not all) of the required Boost libraries were found.  You may
need to install these additional Boost libraries.  Alternatively, set
BOOST\textsubscript{LIBRARYDIR} to the directory containing Boost libraries or BOOST\textsubscript{ROOT}
to the location of Boost.

\begin{itemize}
\item solution
sudo apt-get install libboost-all-dev
\end{itemize}
\end{itemize}


\begin{itemize}
\item Error
gtest is not found

\begin{itemize}
\item solution
sudo apt install libgtest-dev
\end{itemize}

\item Error
poco was not found

\begin{itemize}
\item solution
sudo apt install libpoco-dev
\end{itemize}

\item Error
Could NOT find TinyXML2 (missing: TinyXML2\textsubscript{LIBRARY} TinyXML2\textsubscript{INCLUDE}\textsubscript{DIR})

\begin{itemize}
\item Solution
sudo apt install libtinyxml2-dev
\end{itemize}

\item Error
ModuleNotFoundError: No module named 'sipconfig'

\begin{itemize}
\item Solution
\url{https://stackoverflow.com/questions/7942887/how-to-configure-pyqt4-for-python-3-in-ubuntu}
↑を参考
自分でsip-4.12.4をビルドする必要があるらしい
\end{itemize}

\item Error
Processing plain cmake package: 'python\textsubscript{orocos}\textsubscript{kdl}'
error: use of deleted function ‘KDL::ChainJntToJacSolver\& KDL::ChainJntToJacSolver::operator=(const KDL::ChainJntToJacSolver\&)’

\begin{itemize}
\item Solution
python\textsubscript{orocos}\textsubscript{kdlのバグによるものだが}、まだ修正されていない。(21/02/03)
参考:\url{https://github.com/ros-melodic-arch/ros-melodic-python-orocos-kdl/issues/2}
orocos-kdlをorocos-kdl-python-gitに置き換える必要がある。

\$ cd \textasciitilde{}/ros/melodic/src/
\$ rm -rf orocos\textsubscript{kinematics}\textsubscript{dynamics}/
\$ git clone \url{https://github.com/orocos/orocos\_kinematics\_dynamics.git}
\$ cd orocos\textsubscript{kinematics}\textsubscript{dynamics}
\$ git checkout bionic\textsubscript{example}
\$ rm -rf pybind11
\$ git clone \url{https://github.com/pybind/pybind11.git}
\end{itemize}

\item Error
FileNotFoundError: The sip directory for PyQt5 could not be located. Please ensure that PyQt5 is installed

\begin{itemize}
\item Solution
sipと同様にsourceからビルド
参考: \url{https://github.com/ros2/ros2/issues/623}
\$ wget \url{https://sourceforge.net/projects/pyqt/files/PyQt5/PyQt-5.11.3/PyQt5\_gpl-5.11.3.tar.gz}
\$ tar xzf PyQt5\textsubscript{gpl}-5.11.3.tar.gz
\$ cd PyQt5\textsubscript{gpl}-5.11.3
\$ python ./configure.py --sip \emph{home/itolab-chotaro}.pyenv/versions/roswork/bin/sip
\$ make -j4
\$ make install

(もしも)
Error: PyQt5 requires Qt v5.0 or later. You seem to be using v4.8.7. Use the --qmake flag to specify the correct version of qmake.
がでたら。
\$ sudo apt install qt5-default

Error: Unable to create the C++ code.
→SIPを上のSIPのURLのバージョンに入れ直す
\end{itemize}
\end{itemize}


\section{std\textsubscript{msgsが使えない}.}
\label{sec:org7218c5a}
\begin{itemize}
\item Error
Makefile:140: recipe for target 'all' failed
make: \textbf{*} [all] Error 2
<== Failed to process package 'std\textsubscript{msgs}': 
  Command '['/home/itolab-chotaro/ros/melodic/install\textsubscript{isolated}/env.sh', 'make', '-j6', '-l6']' returned non-zero exit status 2.

Reproduce this error by running:
==> cd /home/itolab-chotaro/ros/melodic/build\textsubscript{isolated}/std\textsubscript{msgs} \&\& /home/itolab-chotaro/ros/melodic/install\textsubscript{isolated}/env.sh make -j6 -l6

Command failed, exiting.

\item Solution
詳細にエラーを検索した結果, ModuleNotFoundError: No module named 'em' というエラーが原因だと思われる.
\$ pip3 install empy

\item Error
\emph{usr/bin/ld: /home/itolab-chotaro}.pyenv/versions/3.6.5/lib/libpython3.6m.a(ceval.o): relocation R\textsubscript{X86}\textsubscript{64}\textsubscript{PC32} against symbol `\textsubscript{Py}\textsubscript{NoneStruct}' can not be used when making a shared object; recompile with -fPIC
/usr/bin/ld: final link failed: Bad value

\begin{itemize}
\item solution
pyenv の環境の問題
CFLAGS="-fPIC"
CONFIGURE\textsubscript{OPTS}=--enable-shared
つまりは、
env PYTHON\textsubscript{CONFIGURE}\textsubscript{OPTS}="--enable-shared" pyenv install 3.6.8
をpyenv install 前に実行
\end{itemize}
\end{itemize}
\section{ここまでやって以下のコマンドが動作した.}
\label{sec:org136da26}
./src/catkin/bin/catkin\textsubscript{make}\textsubscript{isolated} --install -DCMAKE\textsubscript{BUILD}\textsubscript{TYPE}=Release -DSETUPTOOLS\textsubscript{DEB}\textsubscript{LAYOUT}=OFF -DPYTHON\textsubscript{EXECUTABLE}=/home/itolab-chotaro/.pyenv/versions/3.7.5/bin/python3.7m -DPYTHON\textsubscript{INCLUDE}\textsubscript{DIR}=/home/itolab-chotaro/.pyenv/versions/3.7.5/include/python3.7m


\section{catkin build cv\textsubscript{bridge}}
\label{sec:org19c893f}
\begin{itemize}
\item CMake Warning at /usr/share/cmake-3.10/Modules/FindBoost.cmake:1627 (message):
No header defined for python37; skipping header check
Call Stack (most recent call first):
  CMakeLists.txt:11 (find\textsubscript{package})

\item C++ライブラリをpythonにするため, CMakeがlibboost-pythonを探している.
FindBoost.cmakeの1627行目の前に
message("BOOST DIR is ?" \$\{UPPERCP<PONENT\}) と追加してみた結果.

CMake Error at /usr/share/cmake-3.10/Modules/FindBoost.cmake:1948 (message):
Unable to find the requested Boost libraries.

Boost version: 1.65.1

Boost include path: /usr/include

Could not find the following Boost libraries:

boost\textsubscript{python37}

No Boost libraries were found.  You may need to set BOOST\textsubscript{LIBRARYDIR} to the
directory containing Boost libraries or BOOST\textsubscript{ROOT} to the location of
Boost.
Call Stack (most recent call first):
CMakeLists.txt:11 (find\textsubscript{package})

と出てきた.
これを見るとBoostは/usr/include/boost\textsubscript{python37を探してほしいみたい}.

\begin{itemize}
\item Solution
アズさんにお願いして解決してもらった.
Conan(c/c++のパッケージ管理ツール)を使用してlibboost-devをバージョンアップさせる必要があった.
\end{itemize}
\end{itemize}


\section{rosrun <Tab>}
\label{sec:org4cf52df}
\begin{itemize}
\item error
rosrun rospack: error while loading shared libraries: libpython3.7m.so.1.0: cannot open shared object file: No such file or directory

\item Solution
\$LD\textsubscript{LIBRARY}\textsubscript{PATHにlibpython3.7.so.1.0のPATHが通っていなかった}.

\$ echo \$LD\textsubscript{LIBRARY}\textsubscript{PATH}
> \emph{home/itolab-chotaro}.pyenv/versions/3.7.5/lib/libpython3.7m.so.1.0

\$ vim \textasciitilde{}/.bashrcに以下の行を追加.
"""
export LD\textsubscript{LIBRARY}\textsubscript{PATH}="\emph{home/itolab-chotaro}.pyenv/versions/3.7.5/lib/:\$LD\textsubscript{LIBRARY}\textsubscript{PATH}"
"""

補足:
    \$LD\textsubscript{LIBRARY}\textsubscript{PATH} とは共有ライブラリの場所のPATHを格納.
    共有ライブラリとは, プログラムの実行時に使用されるライブラリの一種
\end{itemize}

\section{roscore}
\label{sec:org5dfdc8d}
\begin{itemize}
\item error
"""
File "/opt/ros/melodic/lib/python2.7/dist-packages/roslaunch/config.py", line 45, in <module>
  import rospkg.distro
File "\emph{home/itolab-chotaro}.pyenv/versions/3.7.5/envs/roswork/lib/python3.7/site-packages/rospkg/distro.py", line 44, in <module>
  import yaml
File "\emph{home/itolab-chotaro}.pyenv/versions/3.7.5/envs/roswork/lib/python3.7/site-packages/yaml/\_\textsubscript{init}\_\textsubscript{.py}", line 399
  class YAMLObject(metaclass=YAMLObjectMetaclass):
                            \^{}
SyntaxError: invalid syntax
"""

\item Solution
roscoreの起動には2.7である必要がある.
しかし, PYTHONPATHでpyenvのパッケージの優先度が高いため, python3.7.5内にあるyamlが起動することで, エラーを出している.

この場合, エラーを起こしている
/opt/ros/melodic/lib/python2.7/dist-packages/roslaunch/config.py
/opt/ros/melodic/lib/python2.7/dist-packages/rosgraph/roslogging.py
の中の最初に以下を追加.
"""
import sys
sys.path.remove('\emph{home/itolab-chotaro}.pyenv/versions/3.7.5/envs/roswork/lib/python3.7/site-packages')
"""
\end{itemize}

\section{rosrun object-detection panorama\textsubscript{detection}\textsubscript{split4.py}}
\label{sec:org141f5ba}
\begin{itemize}
\item error
libgcc\textsubscript{s.so.1} must be installed for pthread\textsubscript{cancel} to work

\item solution
libgcc\textsubscript{s.so.1が存在しないと言われている}.
\$LD\textsubscript{LIBRARY}\textsubscript{PATHにlibgcc}\textsubscript{s.so.1のPATHを調べて追加する必要がある}.

\$ echo \$LD\textsubscript{LIBRARY}\textsubscript{PATH}
> /lib/x86\textsubscript{64}-linux-gnu/libgcc\textsubscript{s.so.1}

vim \textasciitilde{}/py3\textsubscript{ws}/devel\textsubscript{isolated}/setup.bash
\end{itemize}

\section{rosrun object-detection panorama\textsubscript{detection}\textsubscript{split4.py}}
\label{sec:orge2515f2}
\begin{itemize}
\item error
File "/home/itolab-chotaro/py3\textsubscript{ws}/src/object-detection/scripts/img\textsubscript{detection.py}", line 13, in <module>
  from cv\textsubscript{bridge} import CvBridge
ModuleNotFoundError: No module named 'cv\textsubscript{bridge}'

\item cv\textsubscript{bridgeのPYTHONPATHが通っていない}.
\$ vim devel/setup.bashに以下を追加
"""
export PYTHONPATH="/home/itolab-chotaro/cv\textsubscript{bridge}\textsubscript{ws}/devel/lib/python3/dist-packages:\$PYTHONPATH"
"""
\end{itemize}




\section{rosdep install --from-paths src --ignore-src --rosdistro melodic -y}
\label{sec:org78318ce}
\begin{itemize}
\item error
ERROR: the following packages/stacks could not have their rosdep keys resolved
to system dependencies:
message\textsubscript{filters}: No definition of [boost] for OS version []
\begin{verbatim}


\end{verbatim}

\item Solution
rosdep がOSの種類を見分けることができていない
よってOS(18.04の場合はbionic)を追加
\$ rosdep install --from-paths src --ignore-src --rosdistro melodic -y  --os=ubuntu:bionic
\end{itemize}


\begin{itemize}
\item Error
ERROR: the following rosdeps failed to install
 sudo -H apt-get install python-catkin-pkg-modules

\begin{itemize}
\item Solution
rosdep init からやり直して
\$ sudo pip install -U vcstool
\$ vcs import src < melodic-desktop-full.rosinstall
でインストール
\end{itemize}
\end{itemize}

\section{rosrun uvc\textsubscript{camera} uvc\textsubscript{camera}\textsubscript{node}}
\label{sec:orgfad3b17}
\begin{itemize}
\item error
[ WARN] [1611858684.035636779]: Camera calibration file \emph{home/itolab-chotaro}.ros/camera\textsubscript{info}/camera.yaml not found.
opening /dev/video0
terminate called after throwing an instance of 'std::runtime\textsubscript{error}'
  what():  couldn't open /dev/video0
中止

\item Solution
カメラのパーミッションを変更する必要がある。
\$ sudo chmod 777 /dev/video0
\end{itemize}
\end{document}
